\documentclass[8pt,a4paper]{article}
\usepackage[margin=0.45in,top=0.35in,bottom=0.35in]{geometry}
\usepackage{titlesec}
\usepackage{enumitem}
\usepackage{hyperref}
\usepackage{xcolor}
\usepackage{tabularx}
\usepackage{booktabs}
\usepackage{fancyhdr}
\usepackage{graphicx}
\usepackage{float}
\usepackage{multicol}

% Tight spacing
\setlength{\parskip}{0.08em}
\setlength{\parindent}{0pt}
\titlespacing*{\section}{0pt}{0.25em}{0.08em}
\titlespacing*{\subsection}{0pt}{0.15em}{0.05em}
\setlist[itemize]{noitemsep, topsep=0pt, leftmargin=1em, parsep=0pt}

% Colors
\definecolor{primary}{RGB}{44, 82, 130}
\definecolor{secondary}{RGB}{113, 128, 150}

% Header/Footer
\pagestyle{fancy}
\fancyhf{}
\fancyhead[L]{\textcolor{secondary}{\scriptsize Enterprise Onboarding Agent}}
\fancyhead[R]{\textcolor{secondary}{\scriptsize Solution Design}}
\fancyfoot[C]{\scriptsize\thepage}
\renewcommand{\headrulewidth}{0.3pt}

% Section formatting
\titleformat{\section}{\small\bfseries\color{primary}}{\thesection.}{0.3em}{}
\titleformat{\subsection}{\footnotesize\bfseries\color{primary}}{\thesubsection}{0.3em}{}

% Smaller captions
\usepackage[font=scriptsize,labelfont=bf]{caption}
\captionsetup{skip=1pt}

\begin{document}

% Title
\begin{center}
{\large\bfseries\color{primary} Enterprise Customer Onboarding Agent --- Solution Design Document}\\[0.05em]
{\small February 2025 | StackAdapt Case Study Submission}
\end{center}

%==============================================================================
% SECTION A: ARCHITECTURE
%==============================================================================
\section{Architecture Overview}

The solution implements an \textbf{AI-powered automation agent} that orchestrates enterprise customer onboarding from deal closure through SaaS provisioning. Built with \textbf{LangGraph} for state machine orchestration and \textbf{FastAPI} for the REST interface, the agent integrates with multiple enterprise systems, validates business rules, assesses risks using LLM intelligence, and takes autonomous actions including tenant provisioning and task management.

\vspace{0.15em}
\begin{minipage}[t]{0.58\textwidth}
\vspace{0pt}
\subsection{System Integrations}
\vspace{0.1em}
\begin{table}[H]
\scriptsize
\begin{tabularx}{\textwidth}{@{}lX@{}}
\toprule
\textbf{System} & \textbf{Objects \& Key Fields} \\
\midrule
\textbf{Salesforce CRM} & \texttt{Account} (Id, Name, IsDeleted, Status, BillingCountry), \texttt{Opportunity} (StageName, Amount, CloseDate), \texttt{User} (IsActive, Email). \href{https://developer.salesforce.com/docs/atlas.en-us.object_reference.meta/object_reference/sforce_api_objects_list.htm}{\textcolor{blue}{[API Ref]}} \\
\textbf{NetSuite ERP} & \texttt{Invoice} (status, dueDate, amountRemaining, paymentStatus). \href{https://system.netsuite.com/help/helpcenter/en_US/APIs/REST_API_Browser/record/v1/2023.1/index.html}{\textcolor{blue}{[API Ref]}} \\
\textbf{CLM System} & Contract status, signatories, effective dates, key terms. Mock simulating DocuSign CLM. \\
\textbf{Provisioning} & Tenant creation + 14-task onboarding checklist with dependencies and due dates. \\
\bottomrule
\end{tabularx}
\end{table}

\vspace{-0.3em}
{\scriptsize\textbf{Field Selection Rationale:} \texttt{Account.IsDeleted} validates account exists; \texttt{Opportunity.StageName="Closed Won"} confirms deal closure; \texttt{Invoice.status} identifies payment blockers. Minimizes API calls while capturing decision-critical data.}
\end{minipage}
\hfill
\begin{minipage}[t]{0.40\textwidth}
\vspace{0pt}
\begin{figure}[H]
\centering
\includegraphics[width=\textwidth,height=15cm,keepaspectratio]{01_architecture.png}
\caption{Architecture: triggers, LangGraph orchestration, integrations.}
\end{figure}
\end{minipage}

%==============================================================================
% SECTION B: AI AGENT APPLICATION
%==============================================================================
\section{AI Agent Application}

The agent leverages \textbf{OpenAI GPT-4} for intelligent analysis with a deterministic rule-based fallback, ensuring operation even without LLM connectivity.

\vspace{0.1em}
\begin{minipage}[t]{0.56\textwidth}
\vspace{0pt}
\subsection{LLM-Powered Intelligence}
\begin{itemize}
\item \textbf{Risk Analysis}: Evaluates API errors, violations, warnings → risk levels (low/medium/high/critical) with business impact and resolution time.
\item \textbf{Summary Generation}: Human-readable reports for CS teams, translating technical states to actionable insights.
\item \textbf{Action Recommendations}: Prioritized steps with owner assignment (CS, Finance, IT, Legal).
\item \textbf{Error Interpretation}: Converts technical codes (e.g., \texttt{INVALID\_SESSION\_ID}) to plain English with resolution steps.
\end{itemize}

\subsection{Autonomous Actions}
Upon decision: \textbf{(1)} Auto-provisions tenant with tier-appropriate config; \textbf{(2)} Creates 14-task checklist with dependencies/owners; \textbf{(3)} Sends Slack alerts; \textbf{(4)} Emails customer welcome; \textbf{(5)} Records API failures for audit.
\end{minipage}
\hfill
\begin{minipage}[t]{0.42\textwidth}
\vspace{0pt}
\begin{figure}[H]
\centering
\includegraphics[width=\textwidth,height=10cm,keepaspectratio]{02_decision.png}
\caption{Decision: API Errors/Violations → BLOCK, Warnings → ESCALATE, Clear → PROCEED.}
\end{figure}
\end{minipage}

%==============================================================================
% SECTION C: ORCHESTRATION
%==============================================================================
\section{Orchestration \& Event-Driven Flows}

\begin{minipage}[t]{0.50\textwidth}
\vspace{0pt}
\begin{table}[H]
\scriptsize
\begin{tabularx}{\textwidth}{@{}llX@{}}
\toprule
\textbf{Trigger} & \textbf{Source} & \textbf{Endpoint} \\
\midrule
Webhook & Salesforce Flow & \texttt{POST /webhook/onboarding} \\
Manual & CS Team & \texttt{POST /demo/run/\{id\}} \\
Batch & Scheduler & \texttt{POST /demo/run-all} \\
\bottomrule
\end{tabularx}
\end{table}

\vspace{-0.3em}
{\scriptsize\textbf{State Machine}: Init → Fetch(SF/CLM/NS) → Validate → Analyze(LLM) → Decide → [Provision] → Notify → Complete.}
\end{minipage}
\hfill
\begin{minipage}[t]{0.48\textwidth}
\vspace{0pt}
{\scriptsize\textbf{Error Simulation}: \texttt{/demo/enable-random-errors} injects failures (401 auth, 400 validation, 429 rate limit, 500 server) for chaos testing without modifying business logic.}
\end{minipage}

%==============================================================================
% SECTION D: TRADE-OFFS
%==============================================================================
\section{Trade-offs, Assumptions \& Considerations}

\begin{minipage}[t]{0.32\textwidth}
\vspace{0pt}
\subsection{Key Trade-offs}
\begin{table}[H]
\scriptsize
\begin{tabularx}{\textwidth}{@{}lX@{}}
\toprule
\textbf{Decision} & \textbf{Trade-off} \\
\midrule
API Errors → BLOCK & Integrity over speed \\
Sync Processing & Simple but slow \\
Rule-based Fallback & Less smart but reliable \\
14 Fixed Tasks & Predictable but rigid \\
\bottomrule
\end{tabularx}
\end{table}
\end{minipage}
\hfill
\begin{minipage}[t]{0.32\textwidth}
\vspace{0pt}
\subsection{Assumptions}
{\scriptsize
\begin{itemize}
\item Salesforce is source of truth for accounts
\item CLM status reflects actual signatures
\item Invoice payment status is current
\item CS monitors Slack for alerts
\item Single onboarding per account
\end{itemize}
}
\end{minipage}
\hfill
\begin{minipage}[t]{0.32\textwidth}
\vspace{0pt}
\subsection{Limitations}
{\scriptsize
\begin{itemize}
\item Mock integrations (not real APIs)
\item In-memory state (lost on restart)
\item Single instance (no scaling)
\item No retry for transient failures
\item No approval workflow UI
\end{itemize}
}
\end{minipage}

%==============================================================================
% SECTION E: MCP COLLABORATION
%==============================================================================
\section{Multi-Agent Collaboration via Model Context Protocol (MCP)}

\begin{minipage}[t]{0.54\textwidth}
\vspace{0pt}
The architecture supports multi-agent collaboration through MCP, enabling specialized agents to communicate via standardized tool interfaces.

\vspace{0.1em}
\begin{table}[H]
\scriptsize
\begin{tabularx}{\textwidth}{@{}lX@{}}
\toprule
\textbf{Agent} & \textbf{Responsibility \& MCP Tools} \\
\midrule
Coordinator & Orchestrates workflow (\texttt{salesforce.*}, \texttt{provision.*}) \\
Contract Agent & Monitors signatures (\texttt{clm.get\_contract}) \\
Finance Agent & Tracks payments (\texttt{netsuite.get\_invoice}) \\
Task Monitor & Detects overdue tasks (\texttt{tasks.get\_overdue}) \\
\bottomrule
\end{tabularx}
\end{table}

\vspace{-0.3em}
{\scriptsize\textbf{Benefits}: Domain expertise • Shared integrations • Scalable complexity • Cross-agent context sharing.}
\end{minipage}
\hfill
\begin{minipage}[t]{0.44\textwidth}
\vspace{0pt}
\begin{figure}[H]
\centering
\includegraphics[width=\textwidth,height=12cm,keepaspectratio]{03_mcp_architecture.png}
\caption{MCP: Agents collaborate via shared servers wrapping APIs.}
\end{figure}
\end{minipage}

%==============================================================================
% SECTION F: PRODUCTION ROADMAP
%==============================================================================
\section{Production Roadmap}

\begin{minipage}[t]{0.48\textwidth}
\vspace{0pt}

{\small\bfseries\color{primary} Cloud Deployment}

\vspace{0.3em}
{\scriptsize
\textbf{Frontend}: A React-based operations dashboard providing real-time visibility into onboarding runs, decisions, and system health. Enables Customer Success and Operations teams to review onboarding outcomes, inspect violations or API errors, manually intervene when required, and track provisioning and onboarding task completion.

\vspace{0.4em}
\textbf{AWS Infrastructure}: The onboarding agent runs as containerized services on ECS/Fargate, allowing horizontal scaling based on inbound webhook volume. LLM inference is handled via \textbf{AWS Bedrock} rather than direct OpenAI calls, enabling multi-model flexibility (Claude, Llama, Titan), private networking with traffic contained within the VPC, IAM-based authentication, and enterprise-grade security and compliance. Bedrock also allows cost control and model switching without code changes.

\vspace{0.4em}
\textbf{Docker + Kubernetes}: Containerization ensures consistent environments across development, staging, and production. Kubernetes orchestration (or ECS equivalents) enables auto-scaling for traffic spikes (e.g., batch onboarding runs), self-healing services, rolling updates, and zero-downtime deployments as the agent logic evolves.
}
\end{minipage}
\hfill
\begin{minipage}[t]{0.50\textwidth}
\vspace{0pt}

{\small\bfseries\color{primary} CI/CD \& Observability}

\vspace{0.3em}
{\scriptsize
\textbf{CI/CD} (GitHub Actions / CodePipeline): Automated pipelines validate agent logic on every pull request, including unit tests for invariant checks, risk analysis, and decision routing. Deployments are reproducible and auditable, with fast feedback loops, reduced manual error, and the ability to quickly roll back in case of regressions affecting onboarding decisions.

\vspace{0.4em}
\textbf{LangSmith Enterprise}: Provides full observability into LLM-powered components of the onboarding agent, including prompt inputs/outputs, token usage, latency, and cost tracking. Supports prompt versioning, regression detection, and A/B testing of risk analysis and summary generation logic without impacting the deterministic rule-based fallback.

\vspace{0.4em}
\textbf{Monitoring \& Alerting}: Prometheus metrics capture agent throughput, error rates, and decision distributions. Distributed tracing via DataDog or Jaeger enables end-to-end visibility across webhook ingestion, integrations, LLM calls, and provisioning. PagerDuty alerts notify on-call teams of critical failures such as sustained API outages or elevated BLOCK rates.
}
\end{minipage}


\vfill
\begin{center}
\textcolor{secondary}{\scriptsize Enterprise Onboarding Agent | StackAdapt Case Study | Full source code and documentation in attached repository}
\end{center}

\end{document}