\documentclass[9pt,a4paper]{article}
\usepackage[margin=0.5in,top=0.4in,bottom=0.4in]{geometry}
\usepackage{titlesec}
\usepackage{enumitem}
\usepackage{hyperref}
\usepackage{xcolor}
\usepackage{tabularx}
\usepackage{booktabs}
\usepackage{fancyhdr}
\usepackage{graphicx}
\usepackage{float}
\usepackage{multicol}

% Tight spacing
\setlength{\parskip}{0.1em}
\setlength{\parindent}{0pt}
\titlespacing*{\section}{0pt}{0.3em}{0.1em}
\titlespacing*{\subsection}{0pt}{0.2em}{0.05em}
\setlist[itemize]{noitemsep, topsep=0pt, leftmargin=1em, parsep=0pt}

% Colors
\definecolor{primary}{RGB}{44, 82, 130}
\definecolor{secondary}{RGB}{113, 128, 150}

% Header/Footer
\pagestyle{fancy}
\fancyhf{}
\fancyhead[L]{\textcolor{secondary}{\scriptsize Enterprise Onboarding Agent}}
\fancyhead[R]{\textcolor{secondary}{\scriptsize Solution Design}}
\fancyfoot[C]{\scriptsize\thepage}
\renewcommand{\headrulewidth}{0.3pt}

% Section formatting
\titleformat{\section}{\normalsize\bfseries\color{primary}}{\thesection.}{0.3em}{}
\titleformat{\subsection}{\small\bfseries\color{primary}}{\thesubsection}{0.3em}{}

% Smaller captions
\usepackage[font=scriptsize,labelfont=bf]{caption}
\captionsetup{skip=1pt}

\begin{document}

% Title
\begin{center}
{\large\bfseries\color{primary} Enterprise Customer Onboarding Agent --- Solution Design Document}\\[0.05em]
{\small February 2025 | StackAdapt Case Study Submission}
\end{center}

%==============================================================================
% SECTION A: ARCHITECTURE
%==============================================================================
\section{Architecture Overview}

The solution implements an \textbf{AI-powered automation agent} that orchestrates enterprise customer onboarding from deal closure through SaaS provisioning. Built with \textbf{LangGraph} for state machine orchestration and \textbf{FastAPI} for the REST interface, the agent integrates with multiple enterprise systems, validates business rules, assesses risks using LLM intelligence, and takes autonomous actions including tenant provisioning and task management.

\vspace{0.2em}
\begin{minipage}[t]{0.58\textwidth}
\vspace{0pt}
\subsection{System Integrations}
The agent connects to enterprise systems via REST APIs. Field selection focused on business-critical data to minimize payload and optimize validation:

\vspace{0.1em}
\begin{table}[H]
\scriptsize
\begin{tabularx}{\textwidth}{@{}lX@{}}
\toprule
\textbf{System} & \textbf{Objects \& Key Fields} \\
\midrule
\textbf{Salesforce CRM} & \texttt{Account} (Id, Name, IsDeleted, Status, BillingCountry), \texttt{Opportunity} (StageName, Amount, CloseDate), \texttt{User} (IsActive, Email). \href{https://developer.salesforce.com/docs/atlas.en-us.object_reference.meta/object_reference/sforce_api_objects_list.htm}{\textcolor{blue}{[API Ref]}} \\
\textbf{NetSuite ERP} & \texttt{Invoice} (status, dueDate, amountRemaining, paymentStatus). Used for payment verification before provisioning. \href{https://system.netsuite.com/help/helpcenter/en_US/APIs/REST_API_Browser/record/v1/2023.1/index.html}{\textcolor{blue}{[API Ref]}} \\
\textbf{CLM System} & Contract status, signatories, effective dates, key terms. Mock implementation simulating DocuSign CLM. \\
\textbf{Provisioning} & Internal system for tenant creation + 14-task onboarding checklist with dependencies and due dates. \\
\bottomrule
\end{tabularx}
\end{table}

\vspace{-0.2em}
{\scriptsize\textbf{Field Selection Rationale:} \texttt{Account.IsDeleted} validates account exists; \texttt{Opportunity.StageName="Closed Won"} confirms deal closure; \texttt{Invoice.status} identifies payment blockers. This minimizes API calls while capturing decision-critical data.}
\end{minipage}
\hfill
\begin{minipage}[t]{0.40\textwidth}
\vspace{0pt}
\begin{figure}[H]
\centering
\includegraphics[width=\textwidth,height=12cm,keepaspectratio]{01_architecture.png}
\caption{High-level architecture showing trigger sources, LangGraph orchestration, and integration layer.}
\end{figure}
\end{minipage}

%==============================================================================
% SECTION B: AI AGENT APPLICATION
%==============================================================================
\section{AI Agent Application}

The agent leverages \textbf{OpenAI GPT-4} for intelligent analysis with a deterministic rule-based fallback, ensuring the system operates even without LLM connectivity. This dual-mode approach balances intelligence with reliability.

\vspace{0.1em}
\begin{minipage}[t]{0.56\textwidth}
\vspace{0pt}
\subsection{LLM-Powered Intelligence}
\begin{itemize}
\item \textbf{Risk Analysis}: Evaluates API errors, business rule violations, and warnings to generate risk levels (low/medium/high/critical) with business impact assessment and estimated resolution time.
\item \textbf{Summary Generation}: Creates human-readable status reports for Customer Success teams, translating technical states into actionable insights.
\item \textbf{Action Recommendations}: Produces prioritized remediation steps with owner assignment (CS, Finance, IT, Legal) based on issue type and urgency.
\item \textbf{Error Interpretation}: Converts technical API error codes (e.g., \texttt{INVALID\_SESSION\_ID}, \texttt{INSUFFICIENT\_ACCESS}) into plain-English explanations with specific resolution steps.
\end{itemize}

\subsection{Autonomous Actions}
Upon decision, the agent executes: \textbf{(1)} Auto-provisions tenant with tier-appropriate configuration; \textbf{(2)} Creates 14-task onboarding checklist with dependencies, owners, and due dates; \textbf{(3)} Sends Slack alerts to \texttt{\#cs-onboarding} channel; \textbf{(4)} Emails welcome message with login credentials to customer; \textbf{(5)} Records all API failures with full context for audit.
\end{minipage}
\hfill
\begin{minipage}[t]{0.42\textwidth}
\vspace{0pt}
\begin{figure}[H]
\centering
\includegraphics[width=\textwidth,height=8cm,keepaspectratio]{02_decision.png}
\caption{Decision logic: API Errors → BLOCK, Violations → BLOCK, Warnings → ESCALATE, Clear → PROCEED.}
\end{figure}
\end{minipage}

%==============================================================================
% SECTION C: ORCHESTRATION
%==============================================================================
\section{Orchestration \& Event-Driven Flows}

The agent supports multiple trigger mechanisms for different operational needs. LangGraph manages state transitions through a defined workflow with full observability via structured logging and optional LangSmith tracing.

\vspace{0.1em}
\begin{minipage}[t]{0.52\textwidth}
\vspace{0pt}
\begin{table}[H]
\scriptsize
\begin{tabularx}{\textwidth}{@{}llX@{}}
\toprule
\textbf{Trigger} & \textbf{Source} & \textbf{Endpoint} \\
\midrule
Webhook & Salesforce Flow & \texttt{POST /webhook/onboarding} \\
Manual & CS Team & \texttt{POST /demo/run/\{account\_id\}} \\
Batch & Scheduler/Cron & \texttt{POST /demo/run-all} \\
Task Update & CS Action & \texttt{PUT /demo/tasks/\{id\}/\{task\}} \\
\bottomrule
\end{tabularx}
\end{table}

\vspace{-0.2em}
{\scriptsize\textbf{State Machine Flow}: Initialize → Fetch Salesforce → Fetch CLM → Fetch NetSuite → Validate Invariants → Analyze Risks (LLM) → Make Decision → [Provision if PROCEED] → Send Notifications → Generate Summary → Complete. Each node handles errors gracefully and updates state.}
\end{minipage}
\hfill
\begin{minipage}[t]{0.46\textwidth}
\vspace{0pt}
{\scriptsize\textbf{Error Simulation for Resilience Testing}: The endpoint \texttt{/demo/enable-random-errors} injects configurable failures with adjustable rates:
\begin{itemize}
\item Authentication errors (HTTP 401) - expired tokens
\item Validation errors (HTTP 400) - malformed data
\item Rate limit errors (HTTP 429) - API throttling
\item Server errors (HTTP 500) - backend failures
\end{itemize}
This enables chaos testing without modifying business logic, validating the agent's error handling and recovery capabilities.}
\end{minipage}

%==============================================================================
% SECTION D: TRADE-OFFS
%==============================================================================
\section{Trade-offs, Assumptions \& Considerations}

\begin{minipage}[t]{0.48\textwidth}
\vspace{0pt}
\subsection{Design Decisions}
\begin{table}[H]
\scriptsize
\begin{tabularx}{\textwidth}{@{}lX@{}}
\toprule
\textbf{Decision} & \textbf{Rationale} \\
\midrule
API Errors → BLOCK & Data integrity over speed; cannot provision with incomplete data \\
Synchronous Processing & Simplicity for demo; production would use message queues (SQS/RabbitMQ) \\
Rule-based Fallback & Ensures availability without LLM; deterministic behavior for testing \\
14 Fixed Tasks & Predictable workflow; production would use configurable templates \\
\bottomrule
\end{tabularx}
\end{table}
\end{minipage}
\hfill
\begin{minipage}[t]{0.50\textwidth}
\vspace{0pt}
\subsection{Scalability \& Security}
{\scriptsize\textbf{Scalability Path}: Current demo uses synchronous processing with in-memory state. Production deployment would include: message queues for async processing, Redis for distributed state, Kubernetes for horizontal scaling, and LLM response caching for cost optimization.

\textbf{Security \& Governance}: OAuth 2.0 with token refresh simulation, structured audit trails with correlation IDs across all operations, PII masking in logs, role-based access control (RBAC) for API operations, and complete state snapshots for compliance.}
\end{minipage}

%==============================================================================
% SECTION E: MCP COLLABORATION
%==============================================================================
\section{Multi-Agent Collaboration via Model Context Protocol (MCP)}

\begin{minipage}[t]{0.54\textwidth}
\vspace{0pt}
The architecture supports multi-agent collaboration through MCP, enabling specialized agents to communicate via standardized tool interfaces. Each agent is a domain expert with focused responsibilities.

\vspace{0.1em}
\begin{table}[H]
\scriptsize
\begin{tabularx}{\textwidth}{@{}lX@{}}
\toprule
\textbf{Agent} & \textbf{Responsibility \& MCP Tools} \\
\midrule
Coordinator & Orchestrates end-to-end workflow (\texttt{salesforce.*}, \texttt{provision.*}, \texttt{tasks.*}) \\
Contract Agent & Monitors signature status, sends reminders (\texttt{clm.get\_contract}, \texttt{clm.send\_reminder}) \\
Finance Agent & Tracks payments, initiates dunning (\texttt{netsuite.get\_invoice}, \texttt{netsuite.send\_dunning}) \\
Task Monitor & Detects overdue tasks, alerts CS (\texttt{tasks.get\_overdue}, \texttt{notify.alert}) \\
\bottomrule
\end{tabularx}
\end{table}

\vspace{-0.2em}
{\scriptsize\textbf{Benefits}: Separation of concerns (domain experts) • Reusable integrations (shared MCP servers) • Scalable complexity (add agents without modifying existing) • Cross-agent context sharing via tool responses.}
\end{minipage}
\hfill
\begin{minipage}[t]{0.44\textwidth}
\vspace{0pt}
\begin{figure}[H]
\centering
\includegraphics[width=\textwidth,height=8cm,keepaspectratio]{03_mcp_architecture.png}
\caption{MCP architecture: Specialized agents collaborate through shared MCP servers wrapping external APIs.}
\end{figure}
\end{minipage}

\vfill
\begin{center}
\textcolor{secondary}{\scriptsize Enterprise Onboarding Agent | StackAdapt Case Study | Full source code and documentation available in attached repository}
\end{center}

\end{document}