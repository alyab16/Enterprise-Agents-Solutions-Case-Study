\documentclass[8pt,a4paper]{article}
\usepackage[margin=0.45in,top=0.35in,bottom=0.35in]{geometry}
\usepackage{titlesec}
\usepackage{enumitem}
\usepackage{hyperref}
\usepackage{xcolor}
\usepackage{tabularx}
\usepackage{booktabs}
\usepackage{fancyhdr}
\usepackage{graphicx}
\usepackage{float}
\usepackage{multicol}

% Tight spacing
\setlength{\parskip}{0.08em}
\setlength{\parindent}{0pt}
\titlespacing*{\section}{0pt}{0.25em}{0.08em}
\titlespacing*{\subsection}{0pt}{0.15em}{0.05em}
\setlist[itemize]{noitemsep, topsep=0pt, leftmargin=1em, parsep=0pt}

% Colors
\definecolor{primary}{RGB}{44, 82, 130}
\definecolor{secondary}{RGB}{113, 128, 150}

% Header/Footer
\pagestyle{fancy}
\fancyhf{}
\fancyhead[L]{\textcolor{secondary}{\scriptsize Enterprise Onboarding Agent}}
\fancyhead[R]{\textcolor{secondary}{\scriptsize Solution Design}}
\fancyfoot[C]{\scriptsize\thepage}
\renewcommand{\headrulewidth}{0.3pt}

% Section formatting
\titleformat{\section}{\small\bfseries\color{primary}}{\thesection.}{0.3em}{}
\titleformat{\subsection}{\footnotesize\bfseries\color{primary}}{\thesubsection}{0.3em}{}

% Smaller captions
\usepackage[font=scriptsize,labelfont=bf]{caption}
\captionsetup{skip=1pt}

\begin{document}

% Title
\begin{center}
{\large\bfseries\color{primary} Enterprise Customer Onboarding Agent --- Solution Design Document}\\[0.05em]
{\small February 2026 | StackAdapt Case Study Submission}
\end{center}

%==============================================================================
% SECTION A: ARCHITECTURE
%==============================================================================
\section{Architecture Overview}

The solution implements an \textbf{AI-powered automation agent} that orchestrates enterprise customer onboarding from deal closure through SaaS provisioning. Built with \textbf{LangGraph} for state machine orchestration and \textbf{FastAPI} for the REST interface, the agent integrates with multiple enterprise systems, validates business rules, assesses risks using LLM intelligence, and takes autonomous actions including tenant provisioning and task management.

\vspace{0.15em}
\begin{minipage}[t]{0.58\textwidth}
\vspace{0pt}
\subsection{System Integrations}
\vspace{0.1em}
\begin{table}[H]
\scriptsize
\begin{tabularx}{\textwidth}{@{}lX@{}}
\toprule
\textbf{System} & \textbf{Objects \& Key Fields} \\
\midrule
\textbf{Salesforce CRM} & \texttt{Account} (Id, Name, IsDeleted, BillingCountry, Industry, OwnerId), \texttt{Opportunity} (StageName, Amount, CloseDate, AccountId, ContractId), \texttt{User} (IsActive, Email, ProfileId). \href{https://developer.salesforce.com/docs/atlas.en-us.object_reference.meta/object_reference/sforce_api_objects_list.htm}{\textcolor{blue}{[API Ref]}} \\ \\
\textbf{NetSuite ERP} & \texttt{Invoice} (status, dueDate, amountRemaining, total, tranId, entity). \href{https://system.netsuite.com/help/helpcenter/en_US/APIs/REST_API_Browser/record/v1/2023.1/index.html}{\textcolor{blue}{[API Ref]}} \\\\
\textbf{CLM System} & Contract status (DRAFT/SENT/SIGNED/EXECUTED), signatories, effective\_date, expiry\_date, key\_terms. Mock simulating DocuSign CLM. \\\\
\textbf{Provisioning} & Tenant creation + 14-task onboarding checklist with dependencies, owners, and due dates. \\
\bottomrule
\end{tabularx}
\end{table}

\vspace{-0.3em}
{\scriptsize\textbf{Field Selection Rationale:} \texttt{Account.IsDeleted} validates account exists; \texttt{Opportunity.StageName="Closed Won"} confirms deal closure; \texttt{Invoice.status} identifies payment blockers; \texttt{CLM.status} in \{EXECUTED, SIGNED\} required for provisioning. Minimizes API calls while capturing decision-critical data.}

\vspace{1em}
\subsection{Data Flow}
{\scriptsize
\begin{enumerate}[leftmargin=1.5em, itemsep=0pt]
\item Webhook received with \texttt{account\_id} and \texttt{correlation\_id}
\item Sequential fetch: Salesforce $\rightarrow$ CLM $\rightarrow$ NetSuite
\item Invariant validation across 5 domains
\item LLM risk analysis (or rule-based fallback)
\item Decision: PROCEED $\rightarrow$ provision; BLOCK/ESCALATE $\rightarrow$ notify
\item Generate reports and complete
\end{enumerate}
}

\end{minipage}
\hfill
\begin{minipage}[t]{0.40\textwidth}
\vspace{0pt}
\begin{figure}[H]
\centering
\includegraphics[width=\textwidth,height=15cm,keepaspectratio]{01_architecture.png}
\caption{Architecture: triggers, LangGraph orchestration, integrations.}
\end{figure}
\end{minipage}

%==============================================================================
% SECTION B: AI AGENT APPLICATION
%==============================================================================
\section{AI Agent Application}

The agent leverages \textbf{OpenAI GPT-4o-mini} (configurable via \texttt{OPENAI\_MODEL}) for intelligent analysis with a deterministic rule-based fallback, ensuring operation even without LLM connectivity.

\vspace{0.1em}
\begin{minipage}[t]{0.56\textwidth}
\vspace{0pt}
\subsection{LLM-Powered Intelligence}
\begin{itemize}
\item \textbf{Risk Analysis}: Evaluates API errors, violations, warnings $\rightarrow$ risk levels (low/medium/high/critical) with business impact and resolution time estimates.
\item \textbf{Summary Generation}: Human-readable reports for CS teams, translating technical states to actionable insights.
\item \textbf{Action Recommendations}: Prioritized steps with owner assignment (CS, Finance, IT/DevOps, Legal, Sales Ops).
\item \textbf{Error Interpretation}: Converts technical codes (e.g., \texttt{INVALID\_SESSION\_ID}) to plain English with resolution steps and responsible teams.
\end{itemize}

\vspace{8pt}

\subsection{Autonomous Actions}
Upon decision, the agent performs the following autonomous actions:
\begin{enumerate}[label=\textbf{(\arabic*)}, itemsep=0pt, topsep=6pt]
\item Auto-provisions the SaaS tenant with tier-appropriate configuration (Enterprise/Growth/Starter).
\item Creates a 14-task onboarding checklist with dependencies, ownership (system/cs\_team/customer), and due dates.
\item Sends Slack notifications to \texttt{\#cs-onboarding-alerts} (blocked) or \texttt{\#cs-onboarding} (success/escalate).
\item Sends customer-facing welcome emails with tenant ID and login URL.
\item Generates HTML/Markdown/JSON reports for auditing and traceability.
\end{enumerate}
\end{minipage}
\hfill
\begin{minipage}[t]{0.42\textwidth}
\vspace{0pt}
\begin{figure}[H]
\centering
\includegraphics[width=\textwidth,height=10cm,keepaspectratio]{02_decision.png}
\caption{Decision: API Errors/Violations $\rightarrow$ BLOCK, Warnings $\rightarrow$ ESCALATE, Clear $\rightarrow$ PROCEED.}
\end{figure}
\end{minipage}

%-------------------------------
% Onboarding Task Management
%-------------------------------
\vspace{0.5em}
\subsection{Onboarding Task Management}

\vspace{0.25em}
{\scriptsize
The provisioning system creates a 14-task onboarding checklist with the following structure:

\vspace{0.15em}
\begin{minipage}[t]{0.48\textwidth}
\textbf{Task Categories:}
\begin{itemize}[leftmargin=1.2em]
\item \textbf{Automated (4)}: Tenant creation, API credentials, welcome email, training materials
\item \textbf{CS Team (5)}: Schedule/conduct kickoff, SSO config, custom reports, 30-day check-in
\item \textbf{Customer (4)}: Verify login, complete tour, invite team, create first campaign
\item \textbf{Milestone (1)}: Onboarding complete marker
\end{itemize}
\end{minipage}
\hfill
\begin{minipage}[t]{0.48\textwidth}
\textbf{API Endpoints:}
\begin{itemize}[leftmargin=1.2em]
\item \texttt{GET /tasks/\{id\}} -- All tasks with status
\item \texttt{GET /tasks/\{id\}/pending?owner=cs\_team} -- Filtered pending
\item \texttt{GET /tasks/\{id\}/overdue} -- Overdue task alerts
\item \texttt{PUT /tasks/\{id\}/\{task\_id\}} -- Update task status
\item \texttt{GET /tasks/\{id\}/next-actions} -- Ready-to-execute tasks
\end{itemize}
\end{minipage}

\vspace{0.15em}
Tasks include dependencies (e.g., ``Complete Tour'' depends on ``Verify Login''), due dates calculated relative to provisioning, and ownership assignment enabling filtered views per team.
}

%==============================================================================
% SECTION B.5: BUSINESS RULE VALIDATION
%==============================================================================
\vspace{0.4em}
\subsection{Business Rule Validation (Invariants)}
{\scriptsize
The agent enforces a two-tier validation system across five domains. \textbf{Violations} are blocking (cause BLOCK decision); \textbf{Warnings} are non-blocking (cause ESCALATE for human review).
}

\vspace{0.15em}
\begin{table}[H]
\scriptsize
\begin{tabularx}{\textwidth}{@{}llX@{}}
\toprule
\textbf{Domain} & \textbf{Blocking Violations} & \textbf{Non-blocking Warnings} \\
\midrule
\textbf{Account} & Missing Id/Name, IsDeleted=true & Missing BillingCountry, Industry, OwnerId \\
\textbf{Opportunity} & Stage $\neq$ ``Closed Won'', Missing Id/AccountId & Missing Amount, CloseDate, ContractId \\
\textbf{Contract (CLM)} & Status $\notin$ \{EXECUTED, SIGNED\}, Missing contract\_id & No effective/expiry date, pending signatories \\
\textbf{Invoice} & Voided, Cancelled, Missing invoice\_id & Overdue, Pending, $<$50\% paid, missing due\_date \\
\textbf{User} & Inactive, Missing Id/Email/ProfileId & Missing Title, Department, ManagerId \\
\bottomrule
\end{tabularx}
\end{table}

%==============================================================================
% SECTION C: ORCHESTRATION
%==============================================================================
\section{Orchestration \& Event-Driven Flows}

\begin{minipage}[t]{0.50\textwidth}
\vspace{0pt}
\begin{table}[H]
\scriptsize
\begin{tabularx}{\textwidth}{@{}llX@{}}
\toprule
\textbf{Trigger} & \textbf{Source} & \textbf{Endpoint} \\
\midrule
Webhook & Salesforce Flow & \texttt{POST /webhook/onboarding} \\
Manual & CS Team & \texttt{POST /demo/run/\{account\_id\}} \\
Batch & Scheduler & \texttt{POST /demo/run-all} \\
\bottomrule
\end{tabularx}
\end{table}
\end{minipage}
\hfill
\begin{minipage}[t]{0.48\textwidth}
\vspace{0pt}
{\scriptsize\textbf{Error Simulation}: \texttt{/demo/enable-random-errors} injects failures (401 auth, 400 validation, 429 rate limit, 500 server) at configurable rates for chaos testing without modifying business logic.}

\vspace{0.3em}
{\scriptsize\textbf{Report Generation}: Each run produces three output files:
\begin{itemize}[leftmargin=1.2em]
\item \texttt{run\_report\_*.md} -- Full Markdown report with all details
\item \texttt{email\_*.html} -- Professional HTML email (blocked, escalation, success, or welcome)
\item \texttt{audit\_*.json} -- Machine-readable audit log
\end{itemize}
}
\end{minipage}

\vspace{0.4em}
\begin{figure}[H]
    \centering
    \includegraphics[width=\linewidth]{04_state_machine.png}
    \caption{State machine: full lifecycle from webhook ingestion through decision routing to completion.}
    \vspace{-0.6em}
\end{figure}

%==============================================================================
% SECTION C.5: API ERROR HANDLING
%==============================================================================
\vspace{0.3em}
\subsection{API Error Classification \& Handling}
{\scriptsize
Each integration returns structured error payloads with \texttt{error\_type}, \texttt{error\_code}, \texttt{http\_status}, \texttt{message}, \texttt{resolution}, and \texttt{owner}. Errors are classified and handled as follows:
}

\vspace{0.1em}
\begin{table}[H]
\scriptsize
\begin{tabularx}{\textwidth}{@{}llllX@{}}
\toprule
\textbf{Error Type} & \textbf{HTTP} & \textbf{Example Code} & \textbf{Owner} & \textbf{Resolution} \\
\midrule
Authentication & 401 & \texttt{INVALID\_SESSION\_ID} & IT/DevOps & Re-authenticate, refresh token \\
Authorization & 403 & \texttt{INSUFFICIENT\_ACCESS} & System Admin & Check profile/permission sets \\
Validation & 400 & \texttt{REQUIRED\_FIELD\_MISSING} & Data Admin & Fix field values in source system \\
Rate Limit & 429 & \texttt{REQUEST\_LIMIT\_EXCEEDED} & Integration Admin & Implement backoff, increase quota \\
Server Error & 500 & \texttt{SERVER\_ERROR} & Support Team & Check status page, retry later \\
\bottomrule
\end{tabularx}
\end{table}


%==============================================================================
% SECTION D: TRADE-OFFS
%==============================================================================
\section{Trade-offs, Assumptions \& Considerations}

\begin{minipage}[t]{0.32\textwidth}
\vspace{0pt}
\subsection{Key Trade-offs}
\begin{table}[H]
\scriptsize
\begin{tabularx}{\textwidth}{@{}lX@{}}
\toprule
\textbf{Decision} & \textbf{Trade-off} \\
\midrule
API Errors $\rightarrow$ BLOCK & Integrity over speed \\
Sync Processing & Simple but slow \\
Rule-based Fallback & Less smart but reliable \\
14 Fixed Tasks & Predictable but rigid \\
LLM Advisory Only & Auditable but less adaptive \\
\bottomrule
\end{tabularx}
\end{table}
\end{minipage}
\hfill
\begin{minipage}[t]{0.32\textwidth}
\subsection{Assumptions}
{\scriptsize
\vspace{10pt}
\begin{itemize}
\item Salesforce is source of truth for accounts
\item CLM status reflects actual signatures
\item Invoice payment status is current
\item All invoices are single-currency (CAD) with no partial payments or discounts or credit memos
\item CS monitors Slack channels for alerts
\item Single onboarding per account at a time
\item Customer email available in CLM signatories
\item Account ID is consistent across all systems 
\end{itemize}
}
\end{minipage}
\hfill
\begin{minipage}[t]{0.32\textwidth}
{\scriptsize
\subsection{Limitations}
{\scriptsize
\vspace{10pt}
\begin{itemize}
\item \textbf{Mock integrations}: Salesforce, NetSuite, and CLM use simulated data; not connected to real APIs
\item \textbf{In-memory state}: Provisioning and task data lost on restart; no persistent database
\item \textbf{Single instance}: No horizontal scaling or load balancing for high-volume webhooks
\item \textbf{No approval UI}: ESCALATE decisions notify Slack but lack a UI for human approve/reject workflow
\item \textbf{No webhook auth}: Inbound webhooks not verified via HMAC signature or shared secret
\end{itemize}
}
}
\end{minipage}

%==============================================================================
% SECTION D.5: DEMO SCENARIOS
%==============================================================================
\vspace{0.3em}
\subsection{Testing \& Demonstration}
{\scriptsize
The system includes 5 pre-configured demo scenarios validating all decision paths:
}

\vspace{0.1em}
\begin{table}[H]
\scriptsize
\begin{tabularx}{\textwidth}{@{}llXl@{}}
\toprule
\textbf{Account ID} & \textbf{Scenario} & \textbf{Description} & \textbf{Decision} \\
\midrule
\texttt{ACME-001} & Happy Path & All checks pass, contract executed, invoice paid & PROCEED \\
\texttt{BETA-002} & Opportunity Blocked & Opportunity in ``Negotiation'' stage & BLOCK \\
\texttt{GAMMA-003} & Overdue Invoice & Invoice 25+ days overdue, needs finance review & ESCALATE \\
\texttt{DELETED-004} & Deleted Account & Account marked \texttt{IsDeleted=true} & BLOCK \\
\texttt{MISSING-999} & Account Not Found & Account does not exist in any system & BLOCK \\
\bottomrule
\end{tabularx}
\end{table}

{\scriptsize
\textbf{Standalone Runner}: \texttt{demo\_standalone.py} provides a zero-dependency demonstration of core logic without LangGraph/FastAPI.
}

%==============================================================================
% SECTION E: MCP COLLABORATION (PROPOSED)
%==============================================================================
\section{Multi-Agent Collaboration via Model Context Protocol (MCP) \textit{--- Proposed Extension}}

\begin{minipage}[t]{0.54\textwidth}
\vspace{0pt}
{\scriptsize\textit{Note: This section describes a proposed architectural extension. The current implementation uses direct function calls within a single agent.}}

\vspace{0.15em}
{\scriptsize
The architecture is designed to support multi-agent collaboration through the \textbf{Model Context Protocol (MCP)}, enabling specialized agents to interact through standardized, governed tool interfaces rather than direct API access. This design cleanly separates decision logic from execution concerns while allowing agents to remain focused on their respective domains.
}

\vspace{0.15em}
\begin{table}[H]
\scriptsize
\begin{tabularx}{\textwidth}{@{}lX@{}}
\toprule
\textbf{Agent} & \textbf{Responsibility \& MCP Tools} \\
\midrule
Coordinator & Orchestrates end-to-end workflow (\texttt{salesforce.*}, \texttt{provision.*}) \\
Contract Agent & Monitors contract execution and signatures (\texttt{clm.get\_contract}) \\
Finance Agent & Evaluates billing and payment state (\texttt{netsuite.get\_invoice}) \\
Task Monitor & Detects overdue onboarding tasks (\texttt{tasks.get\_overdue}) \\
\bottomrule
\end{tabularx}
\end{table}

\vspace{-0.3em}
{\scriptsize\textbf{Benefits}: Clear domain ownership $\bullet$ Shared integrations $\bullet$ Scalable agent complexity $\bullet$ Cross-agent context sharing.}

\vspace{0.35em}
{\scriptsize
\textbf{Concurrency \& API Execution Strategy}: MCP servers would act as controlled execution layers for external APIs, enabling concurrent and batch-aware request handling. Where supported, batch or composite API calls are leveraged to reduce network overhead and latency (e.g., Salesforce composite reads). If batch execution fails or yields partial results, the system safely falls back to bounded parallel single-entity requests.

Concurrency limits, rate limiting, and circuit breakers would be enforced at the MCP layer to prevent downstream system overload and cascading failures. This approach improves onboarding throughput while preserving idempotency, error isolation, and predictable failure handling across agents.
}
\end{minipage}
\hfill
\begin{minipage}[t]{0.44\textwidth}
\vspace{0pt}
\begin{figure}[H]
\centering
\includegraphics[width=\textwidth,height=12cm,keepaspectratio]{03_mcp_architecture.png}
\caption{MCP: Specialized agents collaborate through shared execution servers wrapping external APIs.}
\end{figure}
\end{minipage}

\vspace{15pt}
%==============================================================================
% SECTION F: PRODUCTION Considerations
%==============================================================================
\section{Production Considerations}

%-------------------------------
% Infrastructure Layer
%-------------------------------
\begin{minipage}[t]{0.48\textwidth}
\vspace{0pt}

{\small\bfseries\color{primary} Cloud Deployment}

\vspace{0.3em}
{\scriptsize
\textbf{Frontend}: A role-based operations interface (e.g., React, ASP.NET, or Node.js-backed UI) providing real-time visibility into onboarding runs, decisions, and system health. Enables Customer Success and Operations teams to review onboarding outcomes, inspect violations or API errors, manually intervene when required, and track provisioning progress and onboarding task completion.

\vspace{0.4em}
\textbf{AWS Infrastructure}: The onboarding agent executes as containerized services on ECS/Fargate, enabling horizontal scaling based on inbound webhook volume. LLM inference is handled via \textbf{AWS Bedrock} rather than direct OpenAI calls, allowing multi-model flexibility (Claude, Llama, Titan), private networking with traffic contained within the VPC, IAM-based authentication, and enterprise-grade security and compliance. Bedrock enables cost controls and model switching without application changes.

\vspace{0.4em}
\textbf{Docker + Kubernetes}: Containerization ensures consistent environments across development, staging, and production. Kubernetes orchestration (or ECS equivalents) provides auto-scaling for traffic spikes (e.g., batch onboarding runs), self-healing services, rolling updates, and zero-downtime deployments as agent logic evolves.
}
\end{minipage}
\hfill
\begin{minipage}[t]{0.50\textwidth}
\vspace{0pt}

{\small\bfseries\color{primary} CI/CD \& Observability}

\vspace{0.3em}
{\scriptsize
\textbf{CI/CD} (GitHub Actions / CodePipeline): Automated pipelines validate agent logic on every pull request, including unit tests for invariant checks, risk analysis, and decision routing. Deployments are reproducible and auditable, with fast feedback loops, reduced manual error, and the ability to roll back safely in case of regressions impacting onboarding outcomes.

\vspace{0.4em}
\textbf{LangSmith Enterprise}: Provides deep observability into LLM-powered components, including prompt inputs/outputs, token usage, latency, and cost tracking. Supports prompt versioning, regression detection, and A/B testing of risk analysis and summary generation logic without affecting deterministic rule-based execution.

\vspace{0.4em}
\textbf{Monitoring \& Alerting}: Prometheus metrics capture agent throughput, API error rates, and decision distributions. Distributed tracing via DataDog or Jaeger enables end-to-end visibility across webhook ingestion, integrations, LLM calls, and provisioning. PagerDuty alerts notify on-call teams of critical failures such as sustained API outages or elevated BLOCK rates.
}
\end{minipage}

%-------------------------------
% Agent Runtime & API Governance
%-------------------------------
\vspace{0.6em}
{\small\bfseries\color{primary} Agentic Execution \& API Governance}

\vspace{0.25em}
{\scriptsize
\textbf{Deterministic Decision Boundaries}: The onboarding agent follows a hybrid execution model in which deterministic business rules define all state transitions and final decisions (\texttt{PROCEED}, \texttt{ESCALATE}, \texttt{BLOCK}). LLM components are strictly advisory and are limited to risk interpretation, summarization, and action recommendations. LLM outputs cannot directly mutate agent state or override invariant checks, ensuring predictable, testable, and auditable outcomes.

\vspace{0.35em}
\textbf{Optimized Data Fetching}: To minimize latency when fetching from Salesforce, CLM, and NetSuite, the agent employs a tiered optimization strategy: \textbf{(1)} Batch API requests where supported (e.g., Salesforce Composite API) to retrieve Account, Opportunity, User, and Contract in a single round-trip; \textbf{(2)} Concurrent multithreaded fetches for systems that don't support batching, with configurable thread pools per integration; \textbf{(3)} Bounded retry logic with exponential backoff (max 3 retries, 1s/2s/4s delays) for transient failures before escalating to permanent failure state.

\vspace{0.35em}
\textbf{Idempotency \& Correlation-Based Execution}: Each onboarding run is keyed by a \texttt{correlation\_id} propagated across webhook ingestion, integrations, logs, reports, and notifications. This enables safe replays, deduplication of duplicate webhook events, and re-entrant execution without risk of duplicate provisioning, notifications, or side effects.

\vspace{0.35em}
\textbf{API Error Classification \& Retry Strategy}: External API failures are classified into retryable (e.g., transient 5xx, rate limiting) and non-retryable (e.g., validation, authorization) categories. Retryable errors follow bounded exponential backoff, while non-retryable errors immediately halt onboarding and surface actionable diagnostics. Circuit breakers prevent cascading failures across dependent systems.

\vspace{0.35em}
\textbf{Rate Limiting \& Backpressure}: Inbound webhook traffic is protected via per-source rate limits and queue-based ingestion to absorb traffic spikes from batch updates or retries. Concurrency limits are enforced per integration to prevent overwhelming Salesforce, CLM, or NetSuite APIs during peak load.

\vspace{0.35em}
\textbf{Agent Versioning \& Safe Rollouts}: Every onboarding run records the agent version, rule set, and prompt version used. This enables full auditability, regression analysis, and controlled rollouts (e.g., canary deployments) of agent logic and prompt updates without impacting in-flight executions.

\vspace{0.35em}
\textbf{LLM Data Governance \& Security}: All data sent to LLMs is explicitly field-filtered and sanitized to prevent leakage of sensitive information. PII is masked prior to inference, and external system credentials are never exposed beyond their integration boundaries. Trust boundaries between systems are strictly enforced at the API layer.

\vspace{0.35em}
\textbf{Multi-Agent Interoperability}: As described in Section 5, wrapping integrations behind MCP servers cleanly separates agent logic from external system access, enabling independent versioning, access control, and audit logging.

\begin{center}
\textcolor{secondary}{\scriptsize Enterprise Onboarding Agent | StackAdapt Case Study | Full source code and documentation in attached repository}
\end{center}



\end{document}